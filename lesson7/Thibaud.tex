\section{Full Information Dynamics}
In this section, we have now access to the full information of the instance.\\
"Full Information" means we know \textbf{every previous steps} which lead to the current position.
\begin{equation}\label{eq:10}
\begin{aligned}
a_i(t + 1) = F_i(\underset{-}{a}(0), \underset{-}{a}(1), ..., \underset{-}{a}(t), U_i(.))
\end{aligned}
\end{equation}
\begin{itemize}
\item $a_i$ is the position for the agent designed by $i$,
\item $t$ is the current step (time),
\item $U$ is the utility for the targeted agent, following the function of utility “$.$”,
\item and $F$ is the function which will compute the best next position, hence the Full Information Dynamics, thanks to knowing all the solutions of all the agents and the function of utility.
\end{itemize}
\section{Oracle-based Dynamic}
In this case, we calculate the next step, with this time using an oracle to calculate the utility of every previous position for every other position but ours:
\begin{equation}\label{eq:11}
\begin{aligned}
a_i(t + 1) = F_i(U_i(., \underset{-}{a_{-i}}(0)), U_i(., \underset{-}{a_{-i}}(1)), ..., U_i(., \underset{-}{a_{-i}}(t)))
\end{aligned}
\end{equation}
Notes: $a_-i$ means all positions $a_j$ for $j != i$\\
In this case, we don't need to know what people are playing, we need to know if I need to play something different and therefore, what will be my payoff if I play such a given position?\\
BR Dynamic falls in this class: the last step $(a_i(0), a_i(1), ..., a_i(t))$ could be limited to: $a_i(t + 1) = F_i(U_i(⋅, a_i(t)))$
\section{Payoff based Dynamic}
Now we consider all those informations:
\begin{equation}\label{eq:12}
\begin{aligned}
V_i(a_i, t) = \frac{\sum_{k = 1}^t U_i(a_i(k)), \underset{-}{a_{-i}}(k) * \mathbb{1}(a_i(k) = a_i)}{t}
\end{aligned}
\end{equation}
According to this function, we know \textbf{what we will get} when we pick this position, but we only know when we \textbf{actually} go on it.\\
So, we can wonder: \textbf{is there other way to learn over time? To finally learn how to go to NE?}\\
\\
Let's start with full information: Is there a way a learn to play to NE?\\
\\
This strategy is called “\textbf{fictitious play}”:\\
"Every player thinks the other players are playing according to a stationary mixed strategy" ← \textbf{This statement is not true!}
\begin{table}
    \centering
	\begin{tabular}{| l | l | l | l | l | l | l | l | l |}
\hline
q(2, R) = & 0 & 0 & 0 & 1/4 & 2/5 & 3/6 & $\rightarrow$ & 1/3\\
q(2, S) = & 1 & 1/2 & 1/3 & 1/4 & 1/5 & 1/6 & $\rightarrow$ & 1/3\\
q(2, P) = & 0 & 1/2 & 2/3 & 2/4 & 2/5 & 2/6 & $\rightarrow$ & 1/3\\ \hline
  &   &   &   &   &   &   & &\\
1 & R & R & S & S & S & P & &\\
2 & S & P & P & R & R & R & &\\
  &   &   &   &   &   &   & &\\ \hline
q(1, R) = & 1 & 1 & 2/3 & 2/4 & 2/5 & 2/6 & $\rightarrow$ & 1/3\\
q(1, S) = & 0 & 0 & 1/3 & 2/4 & 3/5 & 3/6 & $\rightarrow$ & 1/3\\
q(1, P) = & 0 & 0 & 0 & 0 & 0 & 1/6 & $\rightarrow$ & 1/3\\ \hline
	\end{tabular}
	\caption{Example: Rock-Scissor-Paper}\label{tab:a} 
\end{table}\\
See Table 1 below.\\
Over time, we notice the distribution is 1/3.\\
\\
We calculate the probability $q$ with the following equation:
\begin{equation}\label{eq:13}
\begin{aligned}
q_i^j(a_j) = \frac{\sum_{k = 1}^t \mathbb{1}(a_j(k) = a_j))}{t}
\end{aligned}
\end{equation}
Player $a_i$ sees player $j$ is playing its action $a_j$, where $a_j$ belongs to the set $A_j$ of the following possible actions, ${ R, S, P }$.\\
$a_j(n)$ belongs to $A_j$ action $j$ playing at turn $n$.\\
$a_i$ belongs to BR (for every $j$, plays according to $q_i^j$).\\
For potential games, $q_i^j(a_j) \rightarrow q*^j(a_j)$ $\underset{-}{q*^j}_{j=1..n}$ is a NE.\\
Good news: if $\underset{-}{q*^j} {j=1..n}$ is a pure strategy NE, for $t$ large enough, then $\underset{-}{a}(t)$ is a NE.\\
Examples:\\
\begin{table}
	\centering
	\begin{tabular}{| l | l | l |}
\hline
  & A      & B \\ \hline
A & $0, 0$ & $1, 1$\\
B & $1, 1$ & $0, 0$\\ \hline
	\end{tabular}
	\caption{Normal form of a game}\label{tab:b} 
\end{table}
See Table 2 above.\\
\begin{table}
    \centering
	\begin{tabular}{| l | l | l | l | l | l | l |}
\hline
 & 1 & 1/2 & 2/3 & 2/4 & $\rightarrow$ & 1/2\\
 & 0 & 1/2 & 1/3 & 2/4 & $\rightarrow$ & 1/2\\ \hline
R & A & B & A & B & & \\
C & A & B & A & B & & \\ \hline
 & 1 & 1/2 & 2/3 & 2/4 & $\rightarrow$ & 1/2\\
 & 0 & 1/2 & 1/3 & 2/4 & $\rightarrow$ & 1/2\\ \hline
	\end{tabular}
	\caption{Example: Rock-Scissor-Paper with JAFP}\label{tab:c}
\end{table}
See Table 3 below.\\
\\
But the payoff is always zero!\\
\\
In oracle-based game: an estimate\\
\begin{equation}\label{eq:14}
\begin{aligned}
V_i(a_i, t) = \frac{\sum_{k = 1}^t U_i(a_i(k)), \underset{-}{a_{-i}}(k) * \mathbb{1}(a_i(k) = a_i)}{t}
\end{aligned}
\end{equation}\\
it's called the \textbf{Joint Action Fictitious Play}.\\
\section{Joint Action Fictitious Play (JAFP)}\\
Based on the frequential play, you select at every step, you check the following rules: \\
\begin{itemize}
\item if $a_i(t)$ belongs to $argmax_{a_i}(V_i(a_i, t))$ then $a_i(t + 1) = a_i(t)$: "you play it again"
\item else "you toss a coin":
    \begin{itemize}
        \item $\rightarrow$ w.p $\frac{1 - \epsilon}{(|B_i(t)|)}$, then pick $a_i(t + 1)$ in $B_i(t)$
	    \item $\rightarrow$ w.p $\epsilon$: $a_i(t + 1) = a_i(t)$
    \end{itemize}
\end{itemize}
→ $\epsilon$ is called the \textbf{inertia of the player}.\\
\\
POTENTIAL GAME which is also GENERIC (there are no same action which give the same payoff).\\
Joint Action Fictitious Play (JAFP) converges w.p 1 to a NE.\\
\\
At every time, you have:
\begin{itemize}
    \item $a_i(t)$ reference action,
    \item $U_i(t)$ reference utility.
\end{itemize}
With this information, we can build:\\
\begin{itemize}
    \item w.p $1-\epsilon$: $\rightarrow$ $a_i(t + 1) = a_i(t)$ ; $U_i(t + 1) = U_i(a_i(t + 1), \underset{-}{a{-i}}(t+1))$
    \item w.p $\epsilon$: $\rightarrow$ playing a random $a'_i$, which belongs to $A_i$ ; $a_i(t + 1) = a'_i =$
    \begin{itemize}
        \item $(if U_i(a'_i, a_-i(t + 1)) > U_i(t))$ then $a_i(t + 1) = a'_i$   ; $U_i(t + 1) = U_i(a'_i, a_-i(t + 1))$\\
		\item (else)                               then $a_i(t + 1) = a_i(t) ; U_i(t + 1) = U_i(t)$
    \end{itemize}
\end{itemize}\\
For every $p < 1$, there is an unique $\epsilon > 0$, such as
\begin{itemize}
	\item for $t$ large enough:
    \begin{itemize}
		\item $a_i(t)$ is a NE with probability at least $p$.
    \end{itemize}
\end{itemize}
\\
if you want to be sure you are playing NE:\\
you need to pick $p \uparrow 1$ ; $\epsilon \downarrow 0$\\
\\
If you explore less, you take more time\\
All this dynamic guaranty you to reach NE.\\
The worst case is NE can't be reach.\\
\\
Price of Anarchy: the worst possible NE can't be worst than twice the optimum.\\
\\
Another approach is to do Simulated Annealing.\\
$\RightArrow$ Logistic learning.\\